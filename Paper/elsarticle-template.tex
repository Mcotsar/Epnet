\documentclass[review]{elsarticle}

\usepackage{lineno,hyperref}
\modulolinenumbers[5]

\journal{Journal of \LaTeX\ Templates}

%%%%%%%%%%%%%%%%%%%%%%%
%% Elsevier bibliography styles
%%%%%%%%%%%%%%%%%%%%%%%
%% To change the style, put a % in front of the second line of the current style and
%% remove the % from the second line of the style you would like to use.
%%%%%%%%%%%%%%%%%%%%%%%

%% Numbered
%\bibliographystyle{model1-num-names}

%% Numbered without titles
%\bibliographystyle{model1a-num-names}

%% Harvard
%\bibliographystyle{model2-names.bst}\biboptions{authoryear}

%% Vancouver numbered
%\usepackage{numcompress}\bibliographystyle{model3-num-names}

%% Vancouver name/year
%\usepackage{numcompress}\bibliographystyle{model4-names}\biboptions{authoryear}

%% APA style
%\bibliographystyle{model5-names}\biboptions{authoryear}

%% AMA style
%\usepackage{numcompress}\bibliographystyle{model6-num-names}

%% `Elsevier LaTeX' style
\bibliographystyle{elsarticle-num}
%%%%%%%%%%%%%%%%%%%%%%%

\begin{document}

\begin{frontmatter}

\title{Elsevier \LaTeX\ template\tnoteref{mytitlenote}}
\tnotetext[mytitlenote]{Fully documented templates are available in the elsarticle package on \href{http://www.ctan.org/tex-archive/macros/latex/contrib/elsarticle}{CTAN}.}

%% Group authors per affiliation:
\author{Elsevier\fnref{myfootnote}}
\address{Radarweg 29, Amsterdam}
\fntext[myfootnote]{Since 1880.}

%% or include affiliations in footnotes:
\author[mymainaddress,mysecondaryaddress]{Elsevier Inc}
\ead[url]{www.elsevier.com}

\author[mysecondaryaddress]{Global Customer Service\corref{mycorrespondingauthor}}
\cortext[mycorrespondingauthor]{Corresponding author}
\ead{support@elsevier.com}

\address[mymainaddress]{1600 John F Kennedy Boulevard, Philadelphia}
\address[mysecondaryaddress]{360 Park Avenue South, New York}

\begin{abstract}
This template helps you to create a properly formatted \LaTeX\ manuscript.
\end{abstract}

\begin{keyword}
\texttt{elsarticle.cls}\sep \LaTeX\sep Elsevier \sep template
\MSC[2010] 00-01\sep  99-00
\end{keyword}

\end{frontmatter}

\linenumbers

\section{The Elsevier article class}

\paragraph{Installation} If the document class \emph{elsarticle} is not available on your computer, you can download and install the system package \emph{texlive-publishers} (Linux) or install the \LaTeX\ package \emph{elsarticle} using the package manager of your \TeX\ installation, which is typically \TeX\ Live or Mik\TeX.

\paragraph{Usage} Once the package is properly installed, you can use the document class \emph{elsarticle} to create a manuscript. Please make sure that your manuscript follows the guidelines in the Guide for Authors of the relevant journal. It is not necessary to typeset your manuscript in exactly the same way as an article, unless you are submitting to a camera-ready copy (CRC) journal.

\paragraph{Functionality} The Elsevier article class is based on the standard article class and supports almost all of the functionality of that class. In addition, it features commands and options to format the
\begin{itemize}
\item document style
\item baselineskip
\item front matter
\item keywords and MSC codes
\item theorems, definitions and proofs
\item lables of enumerations
\item citation style and labeling.
\end{itemize}

\section{Front matter}

The author names and affiliations could be formatted in two ways:
\begin{enumerate}[(1)]
\item Group the authors per affiliation.
\item Use footnotes to indicate the affiliations.
\end{enumerate}
See the front matter of this document for examples. You are recommended to conform your choice to the journal you are submitting to.


\section{Background}

Our principal case study examines the variation of the amphorae production located in Baetica (currently Andalusia, south Spain). During the Roman Empire, this ancient province was an important support for the production and distribution of the olive oil to the rest of the Empire, from the Ist to the IIIrd centuries. 
For this reason, a large-scale infrastructure of amphorae production was develop in this area with more than 80 workshops currently located along the Guadalquivir river and its tributary Genil. (citar Berni, Remesalin and Enriquito). 

puedo hablar aqui de que eran rios de navegación y se podía elaborar ánforas (Remesalin)
Currently, this area has experimented multiples geographical changes through anthropic action and the dynamic of the rivers (Enrique y Remesal)


The majority of amphorae identified in this area belong to Dressel 20 typology divided into different sub-typologies (Martin-Kilcher bibliografia). This amphora type was used mostly to transport olive oil for around 300 years. It means that olive oil was an important product in the roman empire, being used in different aspect of the roman daily life such as consumption, lighting and hygiene to satisfy the high demand of Roman Empire. (extenderse) 

The important demand is also showed by the fact that amphorae Dressel 20 were identified with several identifications marks about its provenance (remesal and xavi). However, it is not clear the meaning of the stamps: it could be an agent idenfied as a olive oil producers or an agent identified as pottery factory. In any case, this paper will be only focused to the study of evolution of the amphorae. 


\section{Material and methods}

We analyse a dataset of 470 amphorae collected from 5 different workshops excavated. 
The workshops were located in Malpica, Cerro del Belén, Parlamento, Villaseca and Las Delicias. We created a database where were selected 80-90 samples of each pottery workshops. The choice of these workshops was due to two reasons. First, most of the workshops located were not excavated being impossible the study of the archaeological material. Second, the workshops were selected from different spaces in order to analyse the variability depending on the distance of each workshop. 

\subsection{Field methods}


Eight different measurements were taken for each amphorae sample of the 5 workshops studied. Most of them were focus on the rim sherds whose fragments were the most preserved on the sample (explicar también la variabilidad). Handles measurement were excluded from the study because the sample study was low. The measurements were divided into exterior diameter, inside diameter, rim height, rim width, shape width, rim inside height, rim width and protruring rim. Finally, multivariate methods were used to explore these metrical differences. 

In our study, we have only selected three variants according with three centuries (Dressel B: I; Dressel C: I-II; Dressel D: II; Dressel E: III) defined by Berni (bibliografía de Berni) . We excluded the rest of variants from our analysis ....

\subsection{Principal Component Analysis}

PCA (podría poner biblio de Shennan y el de Jollife (2002) de Principal Component Analysis)

We used Principal Component Analysis (PCA) to simplify a large number of variables into a smaller number of variables. This method allows to create a reduced number of "new variables" which contain all the relevant information of the previous variables without losing relevance. The firsts principal components are expressed as the result of the most variance of the all information from the original variables.  
The information is expressed as the result of most variation retained in the first principal components. (Jollife, 2002)
  

In our study, this method allowed us to reduce our dataset with 8 variables as measurement into 2 variables. 

\subsection{Discriminant Linear Analysis} 


The performed results with PCA were analyze with Lineal Discriminant Analyse (LDA). LDA was used to find a combination among them to define the groups as well as possible. In spite of being similar to PCA, LDA allows to identify and discriminate the variables which allow to distinghs each group and know how many variables are necessary. In our case, LDA was used to explore a better separate training set from the results of the most relevant principal components. 

All data were collected and performed in LibreOffice 4.2.8.2 and analysed in R version 3.2.4. statistical language, using packages MASS (completar)


\section{Results}

\subsection{Principal Component Analysis}

Principal Component Analysis

\subsection{Discriminant Analysis}

Discriminant Analysis 


\section{Discussion and Conclusion}

\section{Bibliography styles}

There are various bibliography styles available. You can select the style of your choice in the preamble of this document. These styles are Elsevier styles based on standard styles like Harvard and Vancouver. Please use Bib\TeX\ to generate your bibliography and include DOIs whenever available.

Here are two sample references: \cite{Feynman1963118,Dirac1953888}.

\section*{References}

\bibliography{mybibfile}

\end{document}